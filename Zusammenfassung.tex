\documentclass[12pt,a4paper]{article}
\usepackage[utf8]{inputenc}
\usepackage[german]{babel}
\usepackage[T1]{fontenc}
\usepackage{amsmath}
\usepackage{amsfonts}
\usepackage{amssymb}
\usepackage{graphicx}
\usepackage{float}              
\usepackage{caption}
\usepackage{adjustbox}
\usepackage{lmodern}
\usepackage{fourier}
\usepackage[hidelinks]{hyperref}
\usepackage[left=2cm,right=2cm,top=2cm,bottom=2cm]{geometry}

\captionsetup[figure]{name=Abb.}

\begin{document}

\tableofcontents
\newpage

\section{Grundlagen Datenbankensysteme}

\subsection{Probleme alternativer Programme}

\begin{itemize}
\item Redundanzen \& Inkonsistenzen
\item eingeschränkter Zugriff auf Daten
\item Probleme bei Mehrbenutzerzugriff
\item Datenverluste, Integritäts- \& Sicherheitsprobleme
\item hoher Entwicklungsaufwand
\end{itemize}

\subsection{Was ein Datenbanksystem ist}

\begin{figure}[H]
\begin{minipage}[t]{0.45\textwidth}
\begin{itemize}
\item \textbf{DBS} = Datenbank (DB) \& Datenbankmanagementsystem (DBMS)
\begin{itemize}
\item Datenbank: Sammlung verknüpfter, persistenter Daten (z. B. Uni mit Studierenden, Kursen usw.)
\item DBMS: Software zur Speicherung, Verwaltung \& Zugriff auf diese Daten
\end{itemize}
\item Zweck: Einheitliche Verwaltung, Zugriff, Integrität, Sicherheit, Transaktionen, Wiederherstellung
\end{itemize}
\end{minipage}
\hfill
\begin{minipage}[t]{0.45\textwidth}
\centering
\vspace{5mm}
\adjustbox{valign=c}{\includegraphics[scale=0.35]{pictures/FunkDBS.png}}
\caption{Funktion Datenbanksystem}
\end{minipage}
\end{figure}

\subsection{Codds 12 Regeln (Zweck eines DBS)}

\begin{itemize}
\item einheitliche Datenintegration
\item effizienter Zugriff über Datenbanksprachen
\item Metdatenkatalog
\item versch. Benutzer-/Anwendungssichten
\item Integritätsbedingungen von DBMS sichergestellt
\item Sicherheitsmechanismen für Schutz der Daten
\item Transaktionen, Synchronisation \& Widerherstellung nach BSOD
\item Unterstützung für Ad-hoc Anfragen \& Schnittstellen
\end{itemize}

\subsection{Datenabstraktion \& Sichten}

\begin{itemize}
\item Drei Ebenen:
\begin{enumerate}
\item physical level: Spreicherung \& Strukturen
\item logical level: Datenmodell, Entities \& Beziehungen
\item view level: Benutzer- \& Anwendungssicht
\end{enumerate}
\item Ziel: Abstraktion \& Datenunabhängigkeit
\begin{itemize}
\item Physische Datenunabhängigkeit: Speicher ändern ohne Logikänderung
\item Logische Datenunabhängigkeit: Logik ändern ohne Anwendungscode anzupassen
\end{itemize}
\end{itemize}

\begin{figure}[H]
\centering
\includegraphics[scale=0.5]{pictures/Datensicht.png}
\caption{Sicht auf Daten}
\end{figure}

\subsection{Schemata \& Instanzen}

\begin{itemize}
\item Instanz = Inhalt
\item Schema = Struktur
\begin{itemize}
\item konzeptionell/logisch: Grundlage für Design
\item Sicht (externes Schema): während Anforderungsanalyse definiert
\item physisch: Speicherstrukturen mit Relationen verbunden
\end{itemize}
\end{itemize}

\subsection{Datenmodelle}

\begin{itemize}
\item \textbf{konzeptionell}: Entity Relationship (ER)-Modell, semantisch, objektorientiert
\item \textbf{logisch}: relational, objektrelational, hierarchisch, Netzwerk, XML/JSON
\end{itemize}

\subsection{Datenbanksprachen}

\begin{itemize}
\item \textbf{DDL (Data Definition Language:} \\
definiert Schemata \& Metadaten (z.B. CREATE TABLE)
\item \textbf{DML (Data Manipulation Language):} \\
Zugriff \& Veränderung der Daten (z.B. SELECT, INSERT)
\begin{itemize}
\item prozedural: wie abgefragt
\item deklarativ: was abgefragt (SQL), nicht prozedural
\end{itemize}
\end{itemize}

\subsection{Anwendungsarchitekturen}

\begin{itemize}
\item basiert auf Client-Server-Modell
\item Trennung der Funktionalitäten einer Anwendung:
\begin{itemize}
\item Benutzeroberfläche und Interaktion
\item Anwendungslogik (Business Logic)
\item Datenverwaltungsfunktionalität (Speicherung, Verwaltung, Zugriff)
\end{itemize}
\end{itemize}

\begin{figure}[H]
\centering
\begin{minipage}[t]{0.32\textwidth}
\centering
\includegraphics[width=\textwidth]{pictures/cs.png}
\caption{Client-Server-Modell}
\end{minipage}
\hfill
\begin{minipage}[t]{0.32\textwidth}
\centering
\includegraphics[width=\textwidth]{pictures/2s.png}
\caption{2-schichtige Architektur}
\end{minipage}
\hfill
\begin{minipage}[t]{0.32\textwidth}
\centering
\includegraphics[width=\textwidth]{pictures/3s.png}
\caption{3-schichtige Architektur}
\end{minipage}
\end{figure}

\end{document}